\documentclass{article}


\usepackage{arxiv}

\usepackage[utf8]{inputenc} % allow utf-8 input
\usepackage[T1]{fontenc}    % use 8-bit T1 fonts
\usepackage{hyperref}       % hyperlinks
\usepackage{url}            % simple URL typesetting
\usepackage{booktabs}       % professional-quality tables
\usepackage{amsfonts}       % blackboard math symbols
\usepackage{nicefrac}       % compact symbols for 1/2, etc.
\usepackage{microtype}      % microtypography
\usepackage{lipsum}

\title{SBWCE: A Large Corpus to Train Spanish Word Embeddings}


\author{
  Cristian Cardellino \\
  Facultad de Matem\'atica, Astronom\'ia, F\'isica y Computaci\'on \\
  Universidad Nacional de C\'ordoba \\
  C\'ordoba, Argentina \\
  \texttt{ccardellino@unc.edu.ar} \\
}

\date{}

\begin{document}

\maketitle

\begin{abstract}
  This article presents a new freely available corpus for the Spanish language,
  containing a very large number of words (over 2 billion), that has been used
  to train two sets of word embeddings using different models: Word2Vec's
  \cite{DBLP:journals/corr/abs-1301-3781} Continuous Bag-Of-Words (CBOW) and
  SkipGram model. Besides the base models from Word2Vec, we also present a
  version of the two models using FastText \cite{rehurek_lrec} training
  algorithm, thus having access to more information for the vectors using the
  character based n-grams given by FastText. The corpus has also been
  automatically annotated with lemmas and part-of-speech tags using Spacy
  \cite{spacy2} and is available for download at
  \url{https://crscardellino.github.io/SBWCE/}.
\end{abstract}


% keywords can be removed
\keywords{Spanish Corpus \and Word Embeddings \and Word2Vec \and FastText}


\section{Introduction}

In the last couple of years, word embeddings have become a mainstream tool to
work with in natural language processing (NLP) tasks. Many different
state-of-the-art results have been improved by the adding of word embeddings to
the stack of natural language resources
\cite{Collobert:2011:NLP:1953048.2078186}.

Techniques originally applied to machine vision tasks, such as deep learning
architectures, who where originally too heavy to train with classical NLP
feature engineer have been adapted, with the help of word embeddings, and
reached impressive results in tasks that used to be dominated by complex
feature engineering and shallow models.

Word embeddings provide a way of making the process of feature engineer
lighter, by being trained on unlabeled data. However, there is a challenge when
it comes to training word embeddings, which lies in the amount of data needed
to train such models. Word embeddings work best the more data the model have to
be trained with.

This is not much of a problem with the English language, as a resource such as
the Wikipedia alone has over 3 billion words. The problem of data arises for
the other languages, such as is the case for Spanish, for which the Wikipedia
has 700 million words words. As such, there is a need for gathering real
examples of Spanish sentences, in large quantities, for the embeddings to be
trained on.

For this reason we gathered a corpus from different sources available online,
not only the WikiMedia\footnote{\url{https://dumps.wikimedia.org/}} but other
sources, many of them coming from parallel corpora (specially from the OPUS
project \cite{TIEDEMANN12.463}).

The text data was cleaned and processed in order to make it available for
others to experiment on it and run different kinds of analysis. Also, we used
different text embeddings models in order to train different embeddings for the
Spanish language, which we are also making available online.

\section{Related Work}

\section{Resources Description}

\subsection{Corpora}

\subsection{Tools}

\section{Corpus}

\subsection{Raw Corpus}

\subsection{Tagged Corpus}

\subsection{Clean Corpus}

\section{Embeddings Models}

\subsection{Hyperparameters}

\subsection{Results}

\section{Conclusions and Future Work}

% \begin{equation}
% \xi _{ij}(t)=P(x_{t}=i,x_{t+1}=j|y,v,w;\theta)= {\frac {\alpha _{i}(t)a^{w_t}_{ij}\beta _{j}(t+1)b^{v_{t+1}}_{j}(y_{t+1})}{\sum _{i=1}^{N} \sum _{j=1}^{N} \alpha _{i}(t)a^{w_t}_{ij}\beta _{j}(t+1)b^{v_{t+1}}_{j}(y_{t+1})}}
% \end{equation}

% \begin{figure}
%   \centering
%   \fbox{\rule[-.5cm]{4cm}{4cm} \rule[-.5cm]{4cm}{0cm}}
%   \caption{Sample figure caption.}
%   \label{fig:fig1}
% \end{figure}

% \begin{table}
%  \caption{Sample table title}
%   \centering
%   \begin{tabular}{lll}
%     \toprule
%     \multicolumn{2}{c}{Part}                   \\
%     \cmidrule(r){1-2}
%     Name     & Description     & Size ($\mu$m) \\
%     \midrule
%     Dendrite & Input terminal  & $\sim$100     \\
%     Axon     & Output terminal & $\sim$10      \\
%     Soma     & Cell body       & up to $10^6$  \\
%     \bottomrule
%   \end{tabular}
%   \label{tab:table}
% \end{table}

\bibliographystyle{unsrt}
\bibliography{references}

\end{document}
